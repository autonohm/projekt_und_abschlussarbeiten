\documentclass{ohm_project_description}

\projecttitle{Visuelle Navigation mit einer Drohne}
\projectauthor{Labor für mobile Robotik}
\projectdate{2025}


\usepackage{lipsum}

\begin{document}

\maketitle
\thispagestyle{fancy}

\vspace*{-3cm}
\begin{figure}[h!]
    \centering
    \includegraphics[height=4cm]{example-image-a}
\end{figure} 


\lipsum[1-2]


\section*{Arbeitspakete}
\begin{itemize}[leftmargin=0.5cm]
    \item Arbeitspaket 1
    \item Arbeitspaket 2
    \item Arbeitspaket 3
    \item Arbeitspaket 4
\end{itemize}

\section*{Voraussetzungen}
\begin{itemize}[leftmargin=0.5cm]
    \item Grundkenntnisse in ROS 
    \item Grundkenntnisse in einer höheren Programmiersprache (z.B. Python, C++)
    \item Grundkenntnisse in der Bildverarbeitung
\end{itemize}

\vspace{0.5cm}
Das Thema kann nach Abstimmung als Bachelor- oder Masterarbeit bearbeitet werden, sowie als Projektarbeit. 


\vfill
\textcolor{ohm_red}{\rule{\linewidth}{0.4mm}}
\textbf{\textcolor{ohm_red}{Labor für mobile Robotik}} \\
\begin{tabular}{@{}ll}
\textbf{Betreuer:} & Prof. Dr. Christian Pfitzner \\
\textbf{E-Mail:}   & \href{mailto:christian.pfitzner@th-nuernberg.de}{christian.pfitzner@th-nuernberg.de} \\
\end{tabular}

\end{document}