\documentclass{ohm_project_description}

\projecttitle{Patientenassistenz- \& Unterstützungslotse}
\projectauthor{Labor für mobile Robotik}
\projectdate{2025}


\usepackage{lipsum}

\begin{document}

\maketitle
\thispagestyle{fancy}

%\vspace*{-3cm}
%\begin{figure}[h!]
%    \centering
%    \includegraphics[height=4cm]{example-image-a}
%\end{figure} 


\section*{Hintergrund \& Zielsetzung}
Das Klinikum Nürnberg zählt zu den größten kommunalen Krankenhäusern in Europa und ist kontinuierlich bestrebt, den Patientenservice zu optimieren und Innovation zu fördern. In Kooperation mit der Technischen Hochschule Nürnberg Georg Simon Ohm, unter Projektleitung von Prof. Dr. Stefan May, wird ein autonomer Begleitroboter entwickelt. Dieser soll Patientinnen und Patienten sowie Besucherinnen und Besucher bei ihrer Ankunft empfangen und sie gezielt durch die Magistralen bis zum jeweiligen Zielort begleiten. Neben der praktischen Navigation soll der Roboter interaktive Sprachsteuerung ermöglichen, um Patienten aktiv zu unterstützen und ihnen eine angenehme Orientierung zu bieten. Dabei steht die Sicherheit im Krankenhaus an oberster Stelle: Brandschutzrichtlinien, Arbeitssicherheit, Notfallprotokolle und Fluchtwege werden sorgfältig berücksichtigt.


\section*{Arbeitspakete}
\begin{itemize}[leftmargin=0.5cm]
    \item Hochpräzise Sensorik: Lidar- und Kamerasysteme für sichere Navigation
    \item Sprachsteuerung \& KI: Natürliche Sprachverarbeitung für reibungslose Interaktion
    \item Auswahl / Inbetriebnahme einer Roboterbasisplattform
\end{itemize}

\section*{Voraussetzungen}
\begin{itemize}[leftmargin=0.5cm]
    \item Grundkenntnisse in einer höheren Programmiersprache (z.B. Python, C++)
    \item Grundkenntnisse in ROS 
    \item Grundkenntnisse in der Bildverarbeitung und/oder Sprachverarbeitung
\end{itemize}

\vspace{0.5cm}
Das Thema kann nach Abstimmung als Bachelor- oder Masterarbeit bearbeitet werden, sowie als Projektarbeit. 


\vfill
\textcolor{ohm_red}{\rule{\linewidth}{0.4mm}}
\textbf{\textcolor{ohm_red}{Labor für mobile Robotik}} \\
\begin{tabular}{@{}ll}
\textbf{Betreuer:} & Prof. Dr. Stefan May \\
\textbf{E-Mail:}   & \href{mailto:stefan.may@th-nuernberg.de}{stefan.may@th-nuernberg.de} \\
\end{tabular}

\end{document}